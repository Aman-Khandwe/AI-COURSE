\documentclass{article}

\usepackage[margin=1.5in]{geometry}
\usepackage{graphicx}
\graphicspath{ {images/} }


\title{Gödel's Incompleteness Theorem}
\author{Aman Khandwe, 19111006}
\date{\today}

\begin{document}
\maketitle
\section{Formal systems: completeness, consistency, and effective axiomatization}
The incompleteness theorems of Gödel are two mathematical logic theorems that deal with the provability limits of formal axiomatic theories.The incompleteness theorems apply to formal systems that are complex enough to express fundamental natural-number arithmetic and are consistent and effectively axiomatized.A formal system is said to be axiomatized effectively if its set are recurringly listed.A set of axioms is complete if that statement or its negation is proved by the axioms for any statement in axioms' language.A set of axioms is consistent if there is no statement for which the axioms can prove both the statement and its negation, and inconsistent otherwise.
\section{First incompleteness theorem}
"Any consistent formal system F within which a certain amount of elementary arithmetic can be carried out is incomplete; i.e., there are statements of the language of F which can neither be proved nor disproved in F."
\section{Second incompleteness theorem}
"Assume F is a consistent formalized system which contains elementary arithmetic. Then {\displaystyle F\not \vdash {\text{Cons}}(F)}{\displaystyle F\not \vdash {\text{Cons}}(F)}."
\section{Relationship with computability}
The inaccuracy theorem is strongly connected with several outcomes in recursion theory on undecidable sets.Stephen Cole Kleene (1943) provided evidence of Gödel's theorem of incompleteness with basic computer theory results. One result shows that the problem of stopping is undecidable. No programme can determine correctly if any P programme is used as an input if P will stop when running with a certain input. Kleene demonstrated that the existence of a complete, effective arithmetic system with certain coherence characteristics would lead to a definable contradiction in the stopping problem.
\end{document}
